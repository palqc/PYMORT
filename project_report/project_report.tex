% Advanced Programming 2025 - Project Report
% HEC Lausanne / UNIL
\documentclass[11pt,a4paper]{article}

% Packages
\usepackage[utf8]{inputenc}
\usepackage[T1]{fontenc}
\usepackage[english]{babel}
\usepackage{amsmath,amssymb,amsthm}
\usepackage{graphicx}
\usepackage{xcolor}
\usepackage{listings}
\usepackage{hyperref}
\usepackage[margin=1in]{geometry}
\usepackage{fancyhdr}
\usepackage{float}
\usepackage{caption}
\usepackage{subcaption}
\usepackage{csquotes}
\usepackage{biblatex}
\addbibresource{references.bib} % Create this file for your references

% Code listing settings
\definecolor{codegreen}{rgb}{0,0.6,0}
\definecolor{codegray}{rgb}{0.5,0.5,0.5}
\definecolor{codepurple}{rgb}{0.58,0,0.82}
\definecolor{backcolour}{rgb}{0.95,0.95,0.92}

\lstdefinestyle{pythonstyle}{
    backgroundcolor=\color{backcolour},   
    commentstyle=\color{codegreen},
    keywordstyle=\color{magenta},
    numberstyle=\tiny\color{codegray},
    stringstyle=\color{codepurple},
    basicstyle=\ttfamily\footnotesize,
    breakatwhitespace=false,         
    breaklines=true,                 
    captionpos=b,                    
    keepspaces=true,                 
    numbers=left,                    
    numbersep=5pt,                  
    showspaces=false,                
    showstringspaces=false,
    showtabs=false,                  
    tabsize=2,
    language=Python
}

\lstset{style=pythonstyle}

% Header and footer
\pagestyle{fancy}
\fancyhf{}
\rhead{Advanced Programming 2025}
\lhead{Project Report}
\rfoot{Page \thepage}
\setlength{\headheight}{14pt}

% Title page information - MODIFY THESE
\title{%
    \Large \textbf{Advanced Programming 2025} \\
    \vspace{0.5cm}
    \LARGE \textbf{PyMORT: Longevity Bond Pricing \& Mortality Modeling} \\[0.5em]
    \vspace{0.3cm}
    \large Final Project Report
}
\author{
    Pierre-Antoine Le Quellec \\
    HEC Lausanne – University of Lausanne \\
    \texttt{pierre-antoine.lequellec@unil.ch} \\
    Student ID: 22438071
}
\date{\today}

\begin{document}

\maketitle
\thispagestyle{empty}
\vspace{1cm}
\begin{abstract}
\vspace{0.5cm}
\noindent
\textbf{PyMORT} addresses the growing challenge of \textit{longevity risk}---the financial risk that people live longer than expected, thereby increasing liabilities for pension funds and insurers. It develops a Python-based longevity risk analytics engine and provides an end-to-end engineering pipeline from Human Mortality Database (HMD) data ingestion and model calibration to stochastic projections, risk-neutral calibration, scenario-based pricing, and risk reporting, with modular pipelines for hedging, sensitivities, scenario analysis, and optional Hull--White interest-rate modeling.

\medskip
\noindent
Our methodology combines actuarial modeling and quantitative finance. Mortality dynamics are estimated using the \textbf{Lee-Carter} and \textbf{Cairns-Blake-Dowd (CBD)} models, fitted to mortality data from the \textit{Human Mortality Database}. Future mortality is projected stochastically, incorporating uncertainty and risk-neutral adjustments via a market price of longevity risk parameter. Using these forecasts, we value longevity-linked instruments---longevity bonds, survivor swaps, and mortality forwards---through expected discounted cash flows under the risk-neutral measure.\textbf{Validation} against published benchmarks and reference implementations is used to check calibration and factor dynamics.

\medskip
\noindent
Key results include realistic mortality projections consistent with published studies and internally validated bond prices. The main contribution is a modular, end-to-end software framework that integrates actuarial modeling, risk-neutral valuation, and Monte Carlo simulation into a reproducible, CLI-driven workflow with typing and continuous integration.
\end{abstract}

\vspace{3cm}
\noindent\textbf{Keywords:} Data Science, Python, Machine Learning, Longevity risk, Mortality modeling, Lee--Carter, Cairns--Blake--Dowd, Risk-neutral valuation, Longevity bonds, Survivor swaps, Quantitative finance.
\vspace{-1.5em}

\newpage
{\small
\tableofcontents
}
\newpage

% ================== MAIN CONTENT ==================

\section{Introduction}
\label{sec:introduction}

\subsection{Background and Motivation}

Over the past decades, the continuous increase in life expectancy has significantly reshaped the landscape of pension systems and insurance markets.
This demographic shift introduces a new form of financial uncertainty known as \textbf{longevity risk}---the risk that individuals live longer than anticipated.
While the extension of human life is a social achievement, it poses a financial challenge for institutions responsible for life-long payments such as pension funds, annuity providers, and governments.
The longer beneficiaries live, the higher the cumulative liabilities become.
To hedge against this risk, financial markets have developed a class of \textbf{longevity-linked securities}, whose cashflows depend on realized survival rates.
These instruments create a bridge between actuarial science and financial engineering, enabling the transfer of longevity risk from pension funds to investors.
This complexity highlights the need for robust, reproducible, pipeline-based software systems that integrate actuarial modeling with financial pricing and risk reporting.

\subsection{Problem Statement}

The valuation of longevity-linked securities requires accurate \textbf{models of future mortality dynamics}. 
Mortality data are inherently complex---age-dependent, non-stationary, and affected by medical progress or sudden shocks such as pandemics. 
Designing models that capture both the \textbf{age structure} and the \textbf{temporal evolution} of mortality, while remaining stable and interpretable, remains an open problem in quantitative risk management. 
Furthermore, financial pricing demands translating real-world mortality forecasts into a \textbf{risk-neutral framework}, accounting for the market’s perception of longevity risk. 
Building such a framework from raw demographic data is both statistically and computationally challenging, particularly when integrating mortality modeling, projections, scenario-based pricing, hedging, sensitivities, risk reporting, and interest-rate modeling into a coherent workflow.

\subsection{Objectives and Goals}

The aim of this project is to develop \textbf{PyMORT}, a Python-based integrated actuarial--financial pricing and risk management engine that combines \textbf{actuarial modeling} and \textbf{financial mathematics}. 
More specifically, PyMORT will:
\begin{itemize}
    \item Fit and calibrate stochastic mortality models, including \textit{Lee--Carter} and \textit{Cairns--Blake--Dowd};
    \item Generate stochastic forecasts of mortality and survival probabilities with quantified uncertainty;
    \item Implement a \textbf{risk-neutral valuation} framework for longevity bonds and related derivatives;
    \item Provide pipeline-based scenario pricing, hedging, stress testing, sensitivity analysis, and risk reporting;
    \item Provide a \textbf{command-line interface} with modular components for fitting, forecasting, pricing, and hedging;
    \item Emphasize reproducibility and extensibility through typing and continuous integration.
\end{itemize}
Through these goals, PyMORT aims to serve as both an educational and practical tool for understanding how demographic dynamics translate into financial risk and asset pricing.

\subsection{Report Organization}

The remainder of this report is structured as follows:
\begin{itemize}
    \item \textbf{Section~2 -- Literature Review / Related Work} surveys the main mortality modeling approaches and existing longevity-linked instruments, including the Lee--Carter and Cairns--Blake--Dowd frameworks.
    \item \textbf{Section~3 -- Methodology} describes the datasets used, data preprocessing steps, and the implementation of the stochastic mortality and pricing models within the PYMORT architecture.
        \begin{itemize}
            \item \textbf{3.1 Data Description} outlines the Human Mortality Database and its key variables.
            \item \textbf{3.2 Approach} details the statistical models and valuation methods.
            \item \textbf{3.3 Implementation} presents the system design and major Python components.
        \end{itemize}
    \item \textbf{Section~4 -- Results} reports experimental outcomes, including model calibration, forecast accuracy, and bond pricing results, supported by tables and visualizations.
    \item \textbf{Section~5 -- Discussion} interprets the findings, highlighting the main challenges, limitations, and lessons learned.
    \item \textbf{Section~6 -- Conclusion and Future Work} summarizes the contributions and outlines potential extensions such as multi-population modeling, market calibration, and integration of stochastic interest rates.
\end{itemize}

\section{Literature Review / Related Work}
\label{sec:literature}

Longevity risk research combines stochastic mortality modeling with the design and valuation of
securities whose cashflows depend on survival outcomes. PyMORT is positioned at this
intersection: it implements classical Lee--Carter and Cairns--Blake--Dowd families, cohort-aware
extensions, and a pricing layer for longevity bonds and mortality derivatives. The review below
highlights the strands that motivate these choices and clarifies how the package aligns with
established methods.

\subsection{Stochastic mortality modeling and cohort effects}

Lee and Carter introduced the benchmark factorization of log mortality rates with a single time
index, estimated via SVD and projected as a random walk with drift \cite{lee1992modeling}. The
teaching summary used in this project reiterates the identifiability constraints and SVD strategy
that inform PyMORT's LCM1 implementation \cite{leecarter_summary2017}. Empirical work has shown
the need to capture cohort effects, motivating models that add a birth-year term and interpret
patterns in the Lexis diagram \cite{carstensen2020lexis}. PyMORT's LCM2 and APCM3 modules follow
this line by adding explicit cohort effects and period dynamics on log mortality.

Cairns, Blake, and Dowd propose modeling logit death probabilities at higher ages with two period
factors, capturing both the level and slope of the age profile \cite{cairns2006twofactor}. Their
framework and later index-oriented extensions \cite{chan2014cbdindexes} underpin PyMORT's
CBDM5/M6/M7 implementations, including cohort terms and a quadratic age component for older ages.
This aligns the package with the dominant actuarial approach for pension-age mortality while
preserving interpretability for pricing applications.

\subsection{Smoothing and old-age mortality}

Mortality surfaces are noisy and contain cohort and period irregularities. Smoothing methods that
respect demographic structure are therefore central to robust calibration. Camarda's constrained
P-spline approach provides smooth age--time surfaces while enforcing plausible demographic shapes
\cite{camarda2019smooth}. Dokumentov, Hyndman, and Tickle emphasize that cohort and period ridges
require flexible two-dimensional smoothing to avoid biased forecasts \cite{dokumentov2018two}.
PyMORT's optional CP-spline module follows this literature by fitting smooth log-mortality
surfaces before model selection and projection.

At very old ages, data sparsity motivates parametric tails. Evidence from U.S. old-age data
suggests Gompertz fits can outperform logistic/Kannisto forms in the 80+ range
\cite{gavrilova2015oldage}, while more recent work cautions against over-reliance on the Kannisto
model without checking fit quality \cite{dang2023kannisto}. PyMORT therefore includes
Gompertz-based extrapolation to extend mortality beyond observed ages and to stabilize survival
probabilities used in pricing.

\subsection{Longevity-linked securities and market context}

Longevity risk transfer instruments emerged as a response to the growing mismatch between
pension liabilities and available hedging capacity \cite{jointforum2013longevity}. Blake, Cairns,
and Dowd provide early analysis of longevity bonds and broader mortality-linked securities,
outlining their role in transferring aggregate longevity risk to capital markets
\cite{blake2006living}. Biffis and Blake review the design space of mortality-linked securities
and derivatives, emphasizing the need for transparent index-based payoffs and investor-friendly
structures \cite{biffis2009mortality}. Industry-facing notes on the EIB longevity bond illustrate
the coupon structure based on cohort survival and the practical challenges of demand and pricing
\cite{understanding2006longevity}. Subsequent analyses of the Swiss Re mortality bond and the
Kortis longevity trend bond clarify how payoff design and cohort definition affect risk transfer
and investor appetite \cite{cairns2005pricing,hunt2015modelling}.

Joint modeling of mortality and interest-rate risk has also been emphasized for longevity bonds,
for example via regime-switching models that allow structural shifts in both mortality and rates
\cite{shen2013longevity}. These studies motivate PyMORT's separation of mortality dynamics from
discounting while keeping the architecture open to stochastic-rate extensions in the pricing
layer.

\subsection{Risk-neutral pricing and longevity derivatives}

Risk-neutral valuation of longevity-linked cashflows requires a market price of longevity risk.
Cairns, Blake, and Dowd propose risk-adjusted pricing for longevity bonds using observed market
information and parameter uncertainty \cite{cairns2006twofactor}, while Cui estimates longevity
risk premia using utility-based indifference pricing \cite{cui2007longevity}. PyMORT operationalizes
this strand via an Esscher tilt on mortality-factor innovations and calibration to quoted
instruments, which is consistent with the risk-premium perspective in these papers.

For derivatives, survivor forwards and swaps generalize longevity bond payoffs into swap-like
contracts and highlight hedging applications across maturities \cite{dawson2007survivor}.
q-forwards are positioned as standardized building blocks for longevity hedging and as reference
contracts linked to mortality indices \cite{coughlan2007qforwards}. Barrieu and Veraart show that
q-forward prices are sensitive to model choice and estimation window, underscoring the need for
multiple mortality models and robust calibration workflows \cite{barrieu2014pricing}. PyMORT's
pricing module mirrors this literature by supporting q-forwards, s-forwards, survivor swaps, and
longevity bonds under multiple mortality models with scenario-based valuation.

\section{Methodology}
\label{sec:methodology}

\subsection{Data Description}
PyMORT is calibrated on \textbf{Human Mortality Database (HMD)} period life-table
exports, using the 1x1 (age-by-year) format for France \cite{hmd2025}. The repository
contains the HMD text exports for deaths, exposures, and central death rates
(\path{Deaths_1x1_France.txt}, \path{Exposures_1x1_France.txt},
\path{Mx_1x1_France.txt}) as well as a cleaned Excel version
(\path{data_france.xlsx}) that is consumed by the loader in
\path{src/pymort/lifetables.py}. Each record provides \texttt{Year}, \texttt{Age},
and sex-specific rates (\texttt{Female}, \texttt{Male}, \texttt{Total}), enabling
construction of central death rates $m_{x,t}$ and, when needed, one-year death
probabilities $q_{x,t}$.

For reproducible experiments, the configuration in
\path{configs/pricing-pipeline.yaml} filters the dataset to
ages 60--95 and calendar years 1970--2019 for the \texttt{Total} population. The
age window reflects the empirical focus of CBD-style models on post-60 mortality
and the age range relevant for pension and longevity-linked contracts
\cite{cairns2006twofactor,chan2014cbdindexes}. The calibration window ends in 2019
to avoid the structural break introduced by COVID-19 and to maintain a stable
pre-pandemic training set for long-horizon projections.

Data quality considerations guide preprocessing. HMD period tables include an
open-age group (e.g., 110+), which is removed to keep a rectangular age--year
grid. Missing or extremely small values are handled by numeric coercion,
interpolation, and a strictly positive floor to support log transformations.
At advanced ages, sampling noise and sparse exposures motivate smoothing and
tail treatment; the project therefore supports CP-spline smoothing
\cite{camarda2019smooth} and Gompertz-based old-age extrapolation, consistent with
evidence that Gompertz fits can be preferable in the 80+ range
\cite{gavrilova2015oldage} and with cautions on Kannisto-style tails when data
quality is limited \cite{dang2023kannisto}.

\subsection{Approach}
The methodological workflow follows an actuarial--finance sequence aligned with longevity
risk literature: (i) fit and select mortality models on a historical window,
(ii) generate stochastic projections with parameter and process uncertainty,
(iii) transform to a pricing measure, and (iv) value longevity-linked
cashflows and risk measures. The approach explicitly separates the
real-world mortality dynamics from the pricing layer, which is consistent
with the modeling/valuation split emphasized in longevity pricing studies
\cite{cairns2006twofactor,cui2007longevity}.

\paragraph{Mortality models and transformations.}
Mortality is modeled either on log-central death rates or on logit
probabilities to balance interpretability and fit across ages. The Lee--Carter
family decomposes mortality into age-specific level/sensitivity and a period
index estimated by SVD \cite{lee1992modeling}, while cohort effects are added
in LCM2 and APCM3 \cite{carstensen2020lexis}. For older ages, the
Cairns--Blake--Dowd (CBD) family uses parsimonious level/slope (and curvature
in M7) factors with optional cohort effects to stabilize pension-age pricing
\cite{cairns2006twofactor,chan2014cbdindexes}. Central death rates are mapped to
one-year probabilities and cohort survival is obtained by compounding annual
death rates; formal definitions are in Appendix~\ref{app:math}. For old-age
tails, a Gompertz extrapolation is used when needed, consistent with empirical
findings that Gompertz fits can be appropriate in the 80+ range
\cite{gavrilova2015oldage}.

\paragraph{Lee--Carter vs.\ CBD trade-offs.}
LC and CBD families are compared to balance flexibility and
interpretability. LC is designed for the full age range and achieves lower
in-sample error on the full grid, whereas CBD is tailored to older ages and
often trades some fit for robustness and stability in pension-age pricing
\cite{lee1992modeling,cairns2006twofactor,chan2014cbdindexes}. The comparison in
Table~\ref{tab:lc_cbd_compare} makes these trade-offs explicit.

\begin{table}[H]
\centering
\small
\caption{Lee--Carter vs.\ CBD models in PyMORT (conceptual trade-offs).}
\label{tab:lc_cbd_compare}
\begin{tabular}{p{0.18\textwidth} p{0.36\textwidth} p{0.36\textwidth}}
\hline
\textbf{Criterion} & \textbf{Lee--Carter (LC)} & \textbf{CBD (M5--M7)} \\
\hline
Primary age focus & Broad age range & Pension/older ages (60+) \\
Fit vs.\ robustness & Lower RMSE on full grid; more flexible & More parsimonious; stable at high ages \\
Factor structure & Single period factor (+ cohort) & Level/slope/curvature (+ cohort) \\
Pricing stability & Sensitive to full-age calibration & Stable for pension-age liabilities \\
\hline
\end{tabular}
\end{table}

\paragraph{Smoothing, selection, and diagnostics.}
To mitigate noise on the age--year grid, CP-spline
smoothing on $\log m$ with constrained penalties \cite{camarda2019smooth} and
2D smoothing strategies \cite{dokumentov2018two}. Candidate models
are compared using in-sample RMSE on $\log m$ or $\text{logit}(q)$ and
information criteria (AIC/BIC), alongside explicit time-split backtests. All
diagnostics are computed against the raw surface to avoid overstating fit
quality on smoothed data.

\paragraph{Stochastic projections and uncertainty.}
Period factors evolve as random walks with drift,
\[
k_t = k_{t-1} + \mu + \sigma \varepsilon_t,
\]
with analogous dynamics for CBD factors, following standard practice in
Lee--Carter and CBD forecasting \cite{lee1992modeling,cairns2006twofactor}.
Parameter uncertainty (residual bootstrap, year-block or
cell resampling) is combined with process uncertainty to produce large scenario sets of
future $m_{x,t}$ and $q_{x,t}$.

\paragraph{Risk-neutral transformation and pricing.}
Pricing applies a risk-neutral adjustment of factor drifts via an Esscher tilt,
with $\lambda$ calibrated to observed longevity-linked prices in the
literature \cite{cairns2006twofactor,cui2007longevity}. Expected discounted
cashflows are then computed for longevity bonds, survivor swaps, and mortality
forwards (q- and s-forwards), as well as cohort life annuities, all of which
are standard instruments in the longevity risk literature
\cite{blake2006living,dawson2007survivor,coughlan2007qforwards,barrieu2014pricing}.
Discounting is modeled with flat rates or stochastic interest-rate scenarios;
when stochastic rates are used, a Hull--White one-factor short-rate model
aligns valuation with fixed-income practice in longevity bond studies
\cite{cairns2005pricing,shen2013longevity}. Details of the Hull--White short-rate
model and its calibration are provided in Appendix~\ref{app:hull_white}.

\paragraph{Scenario analysis, hedging, and sensitivities.}
Risk measurement is based on scenario distributions of present values using
quantiles, VaR/CVaR, and distribution moments, a common reporting framework in
longevity risk management \cite{jointforum2013longevity}. Stress testing uses
interpretable longevity shocks to evaluate robustness of prices and hedge
effectiveness. Hedging and sensitivity analysis quantify exposure to mortality
levels, factor uncertainty, and rate shifts through variance-minimizing hedge
structures and bump-based measures, consistent with practice in
mortality-derivative markets \cite{blake2006living}.

\subsection{Implementation}
The implementation organizes the methodology into a modular Python
architecture under \path{src/pymort/}, with explicit data contracts between
modeling, pricing, and risk analysis components.

\paragraph{Core data structures and artifacts.}
The codebase standardizes key objects via dataclasses: \texttt{MortalityScenarioSet}
(mortality paths and metadata), \texttt{InterestRateScenarioSet} (short-rate
paths and discount factors), \texttt{RiskReport} (VaR/CVaR summaries),
\texttt{ScenarioBundle} (stressed scenario collections), and
\texttt{AllSensitivities} (rates and mortality sensitivities). Scenario sets
and rate scenarios are serialized to compressed \texttt{.npz} with metadata,
while fitted models and pipeline outputs can be persisted as pickles via the
CLI for reproducibility and downstream analysis.

\paragraph{Modeling and projection modules.}
The \path{pymort/models/} package implements LCM1/\allowbreak LCM2/\allowbreak APCM3
and CBD M5/\allowbreak M6/\allowbreak M7, along with Gompertz tail extrapolation
and shared utilities. The \path{pymort/analysis/} package houses CP-spline
smoothing, model fitting and selection, validation/backtesting, residual
bootstrap, and projection orchestration. Outputs are standardized into
\texttt{ProjectionResult} and \texttt{MortalityScenarioSet}, so downstream
pricing and risk tools consume consistent arrays (ages, years, $q$ paths,
survival paths) with attached metadata.

\paragraph{Scenario analysis (\texttt{analysis/scenario\_analysis.py}).}
This module provides utilities to apply deterministic shocks to scenario sets
and recompute survival to enforce monotonicity. It exposes
\texttt{apply\_\allowbreak mortality\_\allowbreak shock} with named stress
families (long-life, short-life, pandemic windows, improvement plateaus, and
accelerated improvements), a life-expectancy shift solver, and cohort trend
shocks. It also builds structured bundles (\texttt{ScenarioBundle}) and named
stress maps (\texttt{generate\_\allowbreak stressed\_\allowbreak scenarios})
that feed the CLI and pipeline.

\paragraph{Sensitivity analysis.}
This module implements bump/scale utilities to compute rate, mortality, and
volatility sensitivities on scenario sets. Helpers price instruments on a
common scenario set, freeze ATM strikes for forwards and swaps to avoid moving
targets, and aggregate results into \texttt{AllSensitivities}. The pipeline
reuses calibration caches and common random numbers when rebuilding scenarios,
improving numerical stability for bump-based measures. Implementation resides
in \path{analysis/sensitivities.py}.

\paragraph{Pricing and hedging (\texttt{pricing/} and \texttt{pricing/hedging.py}).}
Pricing modules implement instrument-specific routines for longevity bonds,
survivor swaps, q- and s-forwards, and cohort life annuities on scenario PV
matrices using shared cashflow and discounting utilities. The hedging module
builds hedge weights via least-squares solvers (OLS/ridge/lasso), supports
bounds and multi-horizon cashflow matching, and emits \texttt{HedgeResult} and
\texttt{GreekHedgeResult} diagnostics with residual summaries.

\paragraph{Interest rates (\texttt{interest\_rates/hull\_white.py}).}
Stochastic discounting is supported via a Hull--White short-rate model that
generates interest-rate scenarios and attaches discount factors to mortality
scenarios through pipeline utilities. Details of the model and its calibration
are provided in Appendix~\ref{app:hull_white}.

\paragraph{Pipelines and key functions.}
\path{pymort/pipeline.py} provides end-to-end functions that wire the modules
into reproducible flows:
\small
\begin{itemize}
    \item \textbf{Core pipelines:} \texttt{build\_\allowbreak projection\_\allowbreak pipeline} and
    \texttt{build\_\allowbreak risk\_\allowbreak neutral\_\allowbreak pipeline} for model selection, smoothing,
    bootstrap, projections, and $\lambda$ calibration under $Q$.
    \item \textbf{Pricing and risk utilities:} \texttt{pricing\_\allowbreak pipeline},
    \texttt{sensitivities\_\allowbreak pipeline}, \texttt{risk\_\allowbreak analysis\_\allowbreak pipeline}.
    \item \textbf{Stress and hedging utilities:} \texttt{stress\_\allowbreak testing\_\allowbreak pipeline},
    \texttt{hedging\_\allowbreak pipeline}, \texttt{reporting\_\allowbreak pipeline}.
    \item \textbf{Rate utilities:} \texttt{build\_\allowbreak interest\_\allowbreak rate\_\allowbreak pipeline},
    \texttt{build\_\allowbreak joint\_\allowbreak scenarios},\\
    \texttt{apply\_\allowbreak hull\_\allowbreak white\_\allowbreak discounting}.
\end{itemize}
\normalsize
The design emphasizes explicit data flow between scenario sets, pricing
outputs, and risk reports, while caching calibration objects for reuse across
valuation and sensitivity runs.

\paragraph{Interfaces and reproducibility.}
The CLI (\path{pymort/cli.py}) provides command groups for data preparation,
smoothing, fitting, scenario generation, stress testing, pricing, risk-neutral
calibration, sensitivities, hedging, reporting, plotting, and one-click
pipelines. Commands accept JSON/YAML configurations and seeds, and write
structured outputs to configurable directories. A small Streamlit app mirrors
the CLI pipeline as a pedagogical and validation front-end; it is secondary to
the scripted workflow. Visualization utilities (fan charts and Lexis diagrams)
support inspection of stochastic mortality surfaces and scenario summaries.

\section{Results}
\label{sec:results}

\subsection{Experimental Setup}
Results are generated on the France HMD period tables (Total population) with
ages 60--95 and years 1970--2019 (36 ages $\times$ 50 years, 1800 cells). For
the results reported below, CBD M7 is fit on the full window and projected over
a 40-year horizon with the last observed year included (2019--2059). A residual
bootstrap with $B=40$ and $n_{\text{process}}=50$ produces $N=2{,}000$ scenarios
for tractable sensitivity and hedging analyses. The pipeline configuration
(\path{configs/pricing-pipeline.yaml}) supports larger runs and identical model
choices. Risk-neutral pricing uses an Esscher tilt $\lambda=0.1$ and a flat
discount rate of 2\%. Pricing is illustrated for a 20-year longevity bond
(issue age 65, principal included); hedging and sensitivities focus on a
25-year life annuity (issue age 65) and a 25-year survivor swap (age 70).
All results below are computed from the pipeline run and stored under
\path{outputs/} for reproducibility.

\subsection{Model Calibration, Projections, and Pricing Results}

\paragraph{Calibration accuracy.}
Table~\ref{tab:calibration_rmse} reports in-sample fit errors computed on the
HMD grid. The LC model delivers low log-m RMSE, while the two-factor CBD fit on
logit($q$) shows higher residual error, consistent with its parsimonious
structure for older ages.

\begin{table}[H]
\centering
\caption{In-sample fit errors on the France 1970--2019 HMD grid.}
\label{tab:calibration_rmse}
\begin{tabular}{|l|c|c|}
\hline
\textbf{Model} & \textbf{Scale} & \textbf{RMSE} \\
\hline
Lee--Carter (LCM1) & $\log m_{x,t}$ & 0.023 \\
CBD (M5) & $\text{logit}(q_{x,t})$ & 0.102 \\
\hline
\end{tabular}
\end{table}

\paragraph{Factor trends.}
The LC time index $k_t$ declines over 1970--2019 with an estimated linear
slope of $-0.658$ per year, indicating sustained mortality improvement. For
CBD, $\kappa_{1,t}$ (level) trends downward at approximately $-0.0189$ per
year, while $\kappa_{2,t}$ (slope) trends slightly upward
($+1.70\times10^{-4}$ per year), suggesting a modest steepening of the age
gradient at older ages.

\paragraph{Observed mortality improvement.}
From the raw HMD data, period death rates fall substantially over the sample
window. For example, the 2019 rates are approximately 56.04\% lower at age 65,
62.07\% lower at age 80, and 46.87\% lower at age 90 compared to 1970. These patterns
align with the negative drift inferred in LC/CBD factors and motivate the
random-walk-with-drift projections in the model layer.

\paragraph{Pricing baseline.}
To contextualize stochastic pricing, a deterministic baseline can be computed
by holding 2019 period mortality rates fixed. For a 20-year longevity bond
issued at age 65 (notional 1.0, principal included) and discounted at 2\%,
the period-life survival to year 20 is $S_{65}(20)\approx0.6205$ and the present
value of survival-linked coupons plus principal is approximately 14.41. The
full PyMORT pipeline replaces this deterministic baseline with Q-measure
scenario pricing, where the Esscher tilt $\lambda$ shifts factor drifts and
prices are reported as scenario means with risk metrics (VaR/CVaR) via the
risk-reporting module.

\paragraph{Q-measure pricing and hedging performance.}
Under the Q-measure scenario set, the 20-year longevity bond price is 15.596
and the 25-year annuity liability has a mean PV of 17.663. A minimum-variance
hedge of the annuity using a 25-year longevity bond and a 25-year survivor swap
yields weights $(-1.133,\;-0.155)$ and reduces variance by 98.81\%. The survivor
swap is priced at-the-money, so its PV is near zero, but its covariance still
contributes to variance reduction. Table~\ref{tab:hedge_minvar} summarizes the
resulting risk metrics.

\begin{table}[H]
\centering
\caption{Minimum-variance hedge for a 25-year life annuity (age 65).}
\label{tab:hedge_minvar}
\begin{tabular}{lcc}
\hline
\textbf{Metric} & \textbf{Unhedged} & \textbf{Hedged} \\
\hline
Mean PV & 17.663 & 0.000 \\
Std. dev. & 0.101 & 0.0110 \\
VaR$_{0.99}$ & 17.878 & 0.0237 \\
CVaR$_{0.99}$ & 17.913 & 0.0279 \\
\hline
\end{tabular}
\end{table}

\paragraph{Rate and mortality sensitivities.}
Table~\ref{tab:rate_sens} reports rate sensitivities for the longevity bond and
annuity at a 2\% short rate with a 1~bp bump. The annuity shows a longer
effective duration and higher convexity than the bond, reflecting its longer
cashflow profile. Mortality delta-by-age indicates that a 1\% proportional
increase in $q$ at ages 65, 75, or 85 reduces annuity PV by about 155.1 (price
units). The sigma-scale vega is 0.0 for both instruments in this run because
sensitivities are computed on a fixed scenario set. Table
\ref{tab:mortality_delta} reports the age-specific deltas.

\begin{table}[H]
\centering
\caption{Rate sensitivity metrics (2\% base rate, 1~bp bump; convexity normalized by price).}
\label{tab:rate_sens}
\begin{tabular}{lcccc}
\hline
\textbf{Instrument} & \textbf{Price} & \textbf{Duration} & \textbf{DV01} & \textbf{Convexity} \\
\hline
Longevity bond 20y (age 65) & 15.596 & 9.956 & $-0.0155$ & 134.543 \\
Life annuity 25y (age 65) & 17.663 & 11.571 & $-0.0204$ & 184.813 \\
\hline
\end{tabular}
\end{table}

\begin{table}[H]
\centering
\caption{Mortality delta-by-age for the 25-year life annuity (1\% $q$ bump).}
\label{tab:mortality_delta}
\begin{tabular}{lccc}
\hline
\textbf{Age bumped} & 65 & 75 & 85 \\
\hline
Delta $\partial P / \partial(1+\varepsilon)$ & $-155.152$ & $-155.129$ & $-155.085$ \\
\hline
\end{tabular}
\end{table}

\paragraph{Scenario analysis.}
Deterministic mortality shocks generate coherent stress scenarios. Table
\ref{tab:scenario_shocks} reports percentage price impacts relative to the base
Q-measure prices for the bond and annuity. The long-life shock increases PVs
by about 0.85--0.97\%, while a 10\% mortality deterioration decreases PVs by a
similar magnitude. Pandemic and plateau shocks induce smaller but non-negligible
changes.

\begin{table}[H]
\centering
\caption{Scenario analysis: price impact vs.\ base Q-measure prices.}
\label{tab:scenario_shocks}
\begin{tabular}{lcc}
\hline
\textbf{Shock} & \textbf{Bond 20y} & \textbf{Annuity 25y} \\
\hline
Long-life (10\% lower $q$) & $+0.850\%$ & $+0.969\%$ \\
Short-life (10\% higher $q$) & $-0.840\%$ & $-0.956\%$ \\
Pandemic 2025 (+30\% for 2y) & $-0.315\%$ & $-0.337\%$ \\
Improvement plateau from 2030 & $-0.076\%$ & $-0.176\%$ \\
Accelerated improvement +1\%/yr from 2025 & $+0.156\%$ & $+0.254\%$ \\
\hline
\end{tabular}
\end{table}

\paragraph{Hull--White discounting.}
When Hull--White discounting is enabled ($a=0.10$, $\sigma=1\%$), the 20-year
bond price increases from 15.596 to 15.855 (+1.66\%) relative to flat-rate
discounting. The simulated discount factor at year 20 has mean 0.679 and
standard deviation 0.179, capturing rate uncertainty. Table~\ref{tab:hw}
summarizes the comparison. Model details are provided in
Appendix~\ref{app:hull_white}.

\begin{table}[H]
\centering
\caption{Hull--White discounting impact on the 20-year longevity bond.}
\label{tab:hw}
\begin{tabular}{lcc}
\hline
\textbf{Metric} & \textbf{Flat 2\%} & \textbf{Hull--White} \\
\hline
Bond price & 15.596 & 15.855 \\
DF$_{20}$ mean & 0.670 & 0.679 \\
DF$_{20}$ std & -- & 0.179 \\
\hline
\end{tabular}
\end{table}

\subsection{Visualizations}
PyMORT produces diagnostic plots that support interpretation of calibration
and projections. The \texttt{visualization} module provides Lexis diagrams for
cohort effects and fan charts for mortality and survival distributions. The
pipeline attaches optional Gompertz tail extensions to fan plots to visualize
old-age extrapolations, and scenario outputs can be exported as NPZ for
reproducible plotting and reporting.

\section{Discussion}
\label{sec:discussion}

The discussion targets the project specification by interpreting the results
for longevity risk management rather than restating them. First, in-sample fit
diagnostics on the HMD grid support the credibility of the calibration layer,
which is essential before using the models for pricing and hedging decisions.
Second, the pricing example
on real HMD data makes the magnitude of longevity-linked cashflows explicit and
shows how the Esscher tilt shifts valuation under a market price of risk. Third,
the explicit Lee--Carter vs.\ CBD comparison clarifies a practical trade-off:
LC offers better full-surface fit but can be less stable for pension-age pricing,
whereas CBD’s parsimonious structure improves robustness for older ages. In
practice, this implies that portfolio pricing and hedge design should align the
model choice with the liability age profile rather than rely on a single
goodness-of-fit criterion.

From a risk-management perspective, the results imply actionable controls.
Minimum-variance hedging using longevity bonds and survivor swaps compresses
tail risk dramatically, suggesting that even a small menu of traded instruments
can materially reduce longevity exposure when calibrated to the same scenario
set. Sensitivity analysis links pricing to risk drivers: duration and convexity
identify rate exposure, while mortality deltas by age and volatility vega
explain how survival shifts and factor uncertainty influence present values.
Scenario analysis then stress-tests these conclusions under long-life,
short-life, and pandemic-style shocks, isolating the pricing and hedge impacts
that matter most for capital planning. Together, these components move beyond
static pricing to a coherent longevity risk management workflow.

The implementation’s main strengths are architectural. The codebase is modular
by construction (data, models, projections, pricing, risk, interest rates),
and the end-to-end pipelines offer reproducible experiments via explicit
configs, cached calibration objects, and consistent scenario containers. Strict
typing and CI checks reinforce correctness and maintainability, which is
crucial when extending models or adding new instruments. Importantly, key
risk-management tools are not add-ons: hedging, sensitivities, scenario
analysis, and Hull--White rate integration are first-class modules wired into
the pipeline and CLI, making the system reliable and extensible for research
and applied valuation.

Limitations and assumptions remain material. Mortality dynamics rely on LC/CBD
structures and random-walk-with-drift factors, which may miss regime changes or
structural cohort effects. The Esscher transform provides a tractable
risk-neutral adjustment in an incomplete market, but $\lambda$ is not unique
and depends on sparse quotes and model choice. Parameter uncertainty is
mitigated through bootstrap scenarios, yet structural model risk persists,
especially at the oldest ages where data are thin. When Hull--White
discounting is used, rates and mortality are still treated as independent, so
any dependence is ignored. The code mitigates these risks by separating
calibration from valuation, exposing stress tests and sensitivities, and
making scenario analysis explicit, but these measures do not eliminate model
risk—they instead make it transparent and quantifiable.

\section{Conclusion and Future Work}
\label{sec:conclusion}

\subsection{Summary}
This report delivered a complete, reproducible longevity modeling and pricing
pipeline grounded in the PyMORT codebase and the actuarial literature. Using
HMD France data for ages 60--95 (1970--2019), the implementation calibrates
classic log-mortality and logit-mortality frameworks (Lee--Carter, CBD M5/M6/M7,
and APC), assesses fit using in-sample RMSE diagnostics, and projects mortality
via random-walk dynamics. The pricing layer connects statistical models to
risk-neutral valuation by applying an Esscher tilt to bootstrap-simulated
survival curves, enabling longevity bond, swap, and forward valuations under a
clear market price of risk assumption. Architecturally, the project emphasizes
modularity (data, models, pricing, interest rates), a CLI-driven workflow, and
risk reporting with sensitivities and scenario analysis, yielding a transparent
research pipeline that can be audited and extended.

Empirically, the results confirm sustained mortality improvements in France
across adult ages and show that the main period factors explain a large share
of variation while retaining tractability for pricing. The deterministic
baseline illustrates the material magnitude of longevity-linked cashflows, and
the stochastic pricing module exposes sensitivity to both mortality dynamics
and the market price of longevity risk. Together, these components demonstrate
how a modern actuarial codebase can unify demographic analysis and financial
valuation in a consistent workflow.

\subsection{Future Directions}
\begin{itemize}
    \item \textbf{Modeling extensions:} incorporate coherent multi-population
    frameworks, richer cohort dynamics, and age-period interactions beyond the
    APC/CBD families to reduce structural model risk, especially at older ages.
    \item \textbf{Market calibration:} calibrate the Esscher parameter (or
    alternative pricing kernels) across broader panels of longevity-linked
    quotes, and compare with indifference or utility-based pricing approaches
    in sparse markets, including liquidity and basis risk.
    \item \textbf{Joint dynamics and portfolio risk:} integrate rate--mortality
    dependence and portfolio-level hedging analytics to evaluate joint risk and
    capital allocation under realistic market conditions.
    \item \textbf{Data scope and product breadth:} expand to multiple countries
    and product structures while stress-testing robustness to data quality and
    structural uncertainty.
\end{itemize}

% ================== REFERENCES ==================
\newpage
\section*{References}
\addcontentsline{toc}{section}{References}

% If using biblatex (recommended)
\printbibliography[heading=none]

% ================== APPENDICES ==================
\newpage
\appendix
\section{Mathematical Details of Mortality Models}
\label{app:math}

\subsection{Notation and transformations}
Let $m_{x,t}$ denote the central death rate at age $x$ and calendar year $t$.
The code converts $m_{x,t}$ to a one-year death probability via the standard
approximation
\begin{equation}
q_{x,t} = \frac{m_{x,t}}{1 + 0.5\,m_{x,t}},
\end{equation}
and the logit transform is defined as
\begin{equation}
\operatorname{logit}(q) = \log\left(\frac{q}{1-q}\right).
\end{equation}
For a cohort starting at age $x$, survival along a projected path is computed
as the cumulative product along the time axis,
\begin{equation}
S_x(h) = \prod_{u=1}^{h} \left(1 - q_{x+u-1,\,t_0+u-1}\right),
\end{equation}
which matches the implementation in \path{lifetables.py} and the scenario
container in \path{analysis/scenario.py}.

\subsection{Lee--Carter M1 and LCM2}
The baseline Lee--Carter specification is
\begin{equation}
\log m_{x,t} = a_x + b_x\,k_t + \varepsilon_{x,t},
\end{equation}
with identifiability enforced in code by $\sum_x b_x = 1$ and $\frac{1}{T}\sum_t k_t = 0$.
The implementation fits $a_x$ as the mean log-mortality over $t$, then applies
an SVD to the centered surface to recover $b_x$ and $k_t$, followed by the
normalization above.

The cohort-extended LCM2 model adds a cohort term,
\begin{equation}
\log m_{x,t} = a_x + b_x\,k_t + \gamma_{t-x},
\end{equation}
which is estimated by (i) fitting the classic LC parameters, (ii) computing
log-residuals, (iii) averaging residuals by cohort $c=t-x$, and (iv) centering
$\gamma_c$ to enforce identifiability.

\subsection{APC M3}
The APC M3 model implemented in \texttt{apc\_m3.py} is
\begin{equation}
\log m_{x,t} = \beta_x + \kappa_t + \gamma_{t-x}.
\end{equation}
Estimation proceeds by removing the age effect $\beta_x=\frac{1}{T}\sum_t \log m_{x,t}$,
then fitting period and cohort effects via least squares using dummy variables.
The first period and cohort levels are set to zero (reference categories) to
identify the model.

\subsection{CBD M5, M6, and M7}
For older ages, PyMORT implements CBD variants on the logit scale. The M5 model
is
\begin{equation}
\operatorname{logit}(q_{x,t}) = \kappa^{(1)}_t + \kappa^{(2)}_t\,(x-\bar{x}),
\end{equation}
estimated by OLS each year with $x$ centered at $\bar{x}$. The cohort-extended
M6 model adds
\begin{equation}
\operatorname{logit}(q_{x,t}) = \kappa^{(1)}_t + \kappa^{(2)}_t\,(x-\bar{x}) + \gamma_{t-x},
\end{equation}
where $\gamma_{t-x}$ is estimated as the average of logit residuals by cohort
and centered to zero mean for identifiability. The quadratic M7 model includes
an additional age term,
\begin{equation}
\operatorname{logit}(q_{x,t}) =
\kappa^{(1)}_t
 + \kappa^{(2)}_t\,(x-\bar{x})
 + \kappa^{(3)}_t\left((x-\bar{x})^2 - \sigma_x^2\right)
 + \gamma_{t-x},
\end{equation}
with $\sigma_x^2$ the mean squared deviation of ages. The quadratic term is
centered to keep $\kappa^{(1)}_t$ interpretable as a level factor.

\subsection{Smoothing and tail extrapolation}
The smoothing module fits a penalized B-spline surface to $\log m_{x,t}$ using
CPsplines, with separate spline degrees and difference penalties in age and
time. Forecasts are produced by evaluating the fitted surface on an extended
year grid. For optional tail handling, \texttt{gompertz.py} fits a per-year
Gompertz curve on a high-age window:
\begin{equation}
m_{x,t} = \exp(a_t + b_t\,x),
\end{equation}
and extrapolates $m_{x,t}$ for ages beyond the observed range.

\section{Bootstrap and Projection Algorithm}
\label{app:bootstrap}

\subsection{Residual bootstrap}
Parameter uncertainty is captured by residual bootstrap. For log-m models
(LCM1, LCM2, and APC), residuals are computed on $\log m$; for CBD models, residuals
are computed on $\operatorname{logit}(q)$. Residuals are resampled by calendar
year blocks (the default in code), added back to the fitted surface, and the
model is re-estimated to obtain a new parameter set and drift/volatility
estimates for the period factors.

\subsection{Random-walk dynamics}
Projected period factors follow a random walk with drift,
\begin{equation}
k_t = k_{t-1} + \mu + \sigma\,\varepsilon_t,\qquad \varepsilon_t \sim \mathcal{N}(0,1),
\end{equation}
applied componentwise for CBD factors $(\kappa^{(1)}_t,\kappa^{(2)}_t,
\kappa^{(3)}_t)$.
Common random numbers are supported by passing fixed innovation tensors, which
stabilizes scenario comparisons and risk sensitivities.

\subsection{Scenario assembly (pseudo-code)}
\begin{lstlisting}[language=Python, caption={Bootstrap projection flow in PyMORT}, label={lst:bootstrap}]
# Input: mortality surface m, ages, years
fit base model on observed surface
residuals = log_m - fitted_log_m  # or logit_q - fitted_logit_q

for b in 1..B:
    resampled = resample_residuals(residuals, mode="year_block")
    synthetic_surface = fitted_surface + resampled
    refit model -> params_b, (mu_b, sigma_b)
    for p in 1..n_process:
        simulate factor paths with random walk
        reconstruct q_paths from simulated factors

stack all paths -> scenario set with q_paths, S_paths
\end{lstlisting}

\section{Risk-Neutral Valuation and Pricing Formulas}
\label{app:pricing}

\subsection{Esscher transform}
PyMORT applies an Esscher tilt to Gaussian random-walk increments. For a factor
increment $\Delta k_t \sim \mathcal{N}(\mu_P,\sigma_P^2)$ under $\mathbb{P}$ and
Esscher parameter $\lambda$, the risk-neutral drift is
\begin{equation}
\mu_Q = \mu_P + \lambda\,\sigma_P^2,
\end{equation}
with $\sigma_Q=\sigma_P$. For multi-factor CBD M7, the tilt is applied
componentwise to $(\kappa^{(1)},\kappa^{(2)},\kappa^{(3)})$.

\subsection{Scenario pricing}
For a scenario set of size $N$, discount factors $D_t$ and cashflows
$CF^{(n)}_t$, PyMORT prices by Monte Carlo expectation,
\begin{equation}
P = \frac{1}{N}\sum_{n=1}^{N}\sum_{t=1}^{T} D_t^{(n)}\,CF_t^{(n)}.
\end{equation}
The following instruments are implemented:
\begin{itemize}
    \item \textbf{Longevity bond:} $CF_t = N_0\,S_x(t)$. The optional principal
    payment is $N_0\,S_x(T)$ at maturity.
    \item \textbf{Survivor swap:} for fixed-payer, $CF_t = N_0(S_x(t)-K)$ on
    scheduled dates. The ATM strike used in code is
    $K = \sum_t D_t\,\mathbb{E}[S_x(t)]\,/\,\sum_t D_t$.
    \item \textbf{$q$-forward:} payoff at settlement $T_s$ is
    $N_0(q_{x,T_m}-K)$ with measurement at $T_m$.
    \item \textbf{$s$-forward:} payoff at settlement $T_s$ is
    $N_0(S_{x,T_m}-K)$ with measurement at $T_m$.
    \item \textbf{Cohort life annuity:} cashflows follow $CF_t = N_0\,S_x(t)$
    after any deferral period, optionally with a terminal benefit.
\end{itemize}

\section{Hull--White Interest Rate Model Details}
\label{app:hull_white}

\subsection{Motivation}
Longevity-linked cashflows are long dated, so pricing and risk metrics are
sensitive to the term structure and to rate volatility. A flat discount curve
can understate convexity and tail risk in present values. The Hull--White
short-rate model provides a parsimonious way to generate stochastic discount
factors consistent with an observed initial curve, while remaining compatible
with Monte Carlo pricing.

\subsection{Model specification}
The one-factor Hull--White model assumes a mean-reverting short rate
\begin{equation}
dr_t = \left(\theta(t) - a\,r_t\right)dt + \sigma\,dW_t,
\end{equation}
where $a>0$ controls mean reversion and $\sigma$ controls rate volatility. The
time-dependent drift $\theta(t)$ is chosen to fit the initial term structure,
and the simulated short-rate paths are integrated to produce discount factors
$D_t=\exp\left(-\int_0^t r_s\,ds\right)$ used in scenario pricing.

\subsection{Calibration principle}
Calibration follows the standard term-structure consistency condition: given
an input zero-coupon curve, $\theta(t)$ is selected so that model-implied
zero-coupon prices match the observed curve at $t=0$. This ensures that the
stochastic scenarios are anchored to the market curve without requiring
additional derivative calibration.

\subsection{Implementation in PyMORT}
The rate engine and pipeline hooks map directly to the codebase:
\begin{itemize}
    \item \path{interest_rates/hull_white.py} provides
    \texttt{simulate\_\allowbreak hull\_\allowbreak white\_\allowbreak paths},
    which returns short-rate paths on a fixed time grid, and
    \texttt{build\_\allowbreak interest\_\allowbreak rate\_\allowbreak scenarios},
    which converts these paths into discount-factor arrays and packages them
    into an \texttt{Interest\allowbreak Rate\allowbreak Scenario\allowbreak Set}
    scenario container. Metadata include times, horizons, and scenario count.
    The function
    \texttt{calibrate\_\allowbreak theta\_\allowbreak from\_\allowbreak zero\_\allowbreak curve}
    computes $\theta(t)$ from the supplied curve on the simulation grid.
    \item \path{pipeline.py} exposes
    \texttt{build\_\allowbreak interest\_\allowbreak rate\_\allowbreak pipeline}
    to generate rate scenarios, \texttt{build\_\allowbreak joint\_\allowbreak scenarios}
    to align rate and mortality scenarios in $(N,T)$, and
    \texttt{apply\_\allowbreak hull\_\allowbreak white\_\allowbreak discounting}
    to attach the discount-factor cube to a mortality scenario set so pricing
    functions consume consistent dimensions.
\end{itemize}

\subsection{Limitations}
The current implementation does not calibrate $(a,\sigma)$ to liquid
interest-rate options and does not model correlation between mortality and
interest rates. Parameter uncertainty in the rate process is also not modeled.
These extensions are left for future work, while the existing framework allows
stress tests and sensitivities to quantify the impact of stochastic discounting.

\section{Extended Results Tables}
\label{app:results}

\begin{table}[H]
\centering
\caption{Fit diagnostics on the HMD grid (in-sample RMSE).}
\begin{tabular}{lcc}
\hline
Model & Metric & Value \\
\hline
LC M1 & RMSE of $\log m$ on HMD grid & 0.023 \\
CBD M5 & RMSE of $\operatorname{logit}(q)$ on HMD grid & 0.102 \\
\hline
\end{tabular}
\end{table}

\begin{table}[H]
\centering
\caption{Estimated period-factor drifts (per year).}
\begin{tabular}{lc}
\hline
Factor & Drift \\
\hline
$k_t$ (LC) & $-0.658$ \\
$\kappa^{(1)}_t$ (CBD) & $-0.0189$ \\
$\kappa^{(2)}_t$ (CBD) & $1.70\times 10^{-4}$ \\
\hline
\end{tabular}
\end{table}

\begin{table}[H]
\centering
\caption{Observed mortality improvements in France (1970 to 2019).}
\begin{tabular}{lcc}
\hline
Age & Reduction in $m_{x,t}$ & Approx. percent \\
\hline
65 & $m_{2019}/m_{1970} \approx 0.440$ & $56.04\%$ \\
80 & $m_{2019}/m_{1970} \approx 0.379$ & $62.07\%$ \\
90 & $m_{2019}/m_{1970} \approx 0.531$ & $46.87\%$ \\
\hline
\end{tabular}
\end{table}

\begin{table}[H]
\centering
\caption{Deterministic pricing baseline (2019 mortality, 2\% flat rate).}
\begin{tabular}{lc}
\hline
Metric & Value \\
\hline
$S_{65}(20)$ & $0.6205$ \\
20-year longevity bond PV & $14.41$ \\
\hline
\end{tabular}
\end{table}

\section{Repository Structure and Reproducibility}
\label{app:code}

\subsection{Module layout}
\begin{lstlisting}[language=Python, caption={Core PyMORT package layout}, label={lst:tree}]
src/pymort/
  analysis/          # fitting, bootstrap, projections, reporting
  models/            # LC, CBD, APC, Gompertz implementations
  pricing/           # bonds, swaps, forwards, annuities, hedging
  interest_rates/    # Hull--White scenario generation
  cli.py             # CLI entrypoints
  pipeline.py        # high-level pipeline functions
\end{lstlisting}

\subsection{Reproducible runs}
The CLI exposes one-click pipelines configured by YAML files stored under
\path{configs/}. The pricing and hedge pipelines used in this report can be
executed via:
\begin{lstlisting}[language=bash, caption={Pipeline entrypoints}]
pymort run pricing-pipeline --config configs/pricing-pipeline.yaml
pymort run hedge-pipeline --config configs/hedge-pipeline.yaml
\end{lstlisting}
Outputs are written to the configured \texttt{outdir} (default \texttt{outputs/})
and include fitted parameters, scenario sets, pricing results, and diagnostic
plots compatible with the reporting utilities.

\end{document}
